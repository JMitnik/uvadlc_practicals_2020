\documentclass{article}

% if you need to pass options to natbib, use, e.g.:
%     \PassOptionsToPackage{numbers, compress}{natbib}
% before loading neurips_2018

% ready for submission
% \usepackage{neurips_2018}

% to compile a preprint version, e.g., for submission to arXiv, add add the
% [preprint] option:
%     \usepackage[preprint]{neurips_2018}

% to compile a camera-ready version, add the [final] option, e.g.:
\usepackage[final]{main}

% to avoid loading the natbib package, add option nonatbib:
%     \usepackage[nonatbib]{neurips_2018}

\usepackage[utf8]{inputenc} % allow utf-8 input
\usepackage[T1]{fontenc}    % use 8-bit T1 fonts
\usepackage{hyperref}       % hyperlinks
\usepackage{url}            % simple URL typesetting
\usepackage{booktabs}       % professional-quality tables
\usepackage{amsmath}       % general math
\usepackage{amsfonts}       % blackboard math symbols
\usepackage{nicefrac}       % compact symbols for 1/2, etc.
\usepackage{microtype}      % microtypography

\title{DL: Assignment 1}

% The \author macro works with any number of authors. There are two commands
% used to separate the names and addresses of multiple authors: \And and \AND.
%
% Using \And between authors leaves it to LaTeX to determine where to break the
% lines. Using \AND forces a line break at that point. So, if LaTeX puts 3 of 4
% authors names on the first line, and the last on the second line, try using
% \AND instead of \And before the third author name.

\author{%
  Jonathan Mitnik \\
  MSc Artificial Intelligence\\
  University of Amsterdam\\
  \texttt{jonathan@student.uva.nl} \\
}

\begin{document}
% \nipsfinalcopy is no longer used
\maketitle

\begin{abstract}
    In this report, the main focus is to go through the assignment, starting with a rough Numpy
    implementation, and then using State-of-the-art PyTorch.
\end{abstract}

\section{Derivatives}
Test123asdasdasdasdasd
asdasdasd

\section{PyTorch Implementation}
Test123asdasdasdasdasd
asdasdasd

\section{Layer normalization}
\subsection*{3.2a: Manual implementation of backward pass}
I will start by writing down, in essence, all known shapes.

\begin{enumerate}
    \item $\emph{\textbf{X}} \in \mathbb{R}^{S * M}$
    \item $\emph{\textbf{Y}} \in \mathbb{R}^{S * M}$ (same shape as X)
    \item $L \in \mathbb{R}^{1}$
    \item $\gamma \in \mathbb{R}^{M}$
    \item $\beta \in \mathbb{R}^{M}$
    \item $\frac{\delta L}{\delta \gamma} \in \mathbb{R}^{M}$
    \item $\frac{\delta L}{\delta \beta} \in \mathbb{R}^{M}$
    \item $\frac{\delta L}{\delta \emph{\textbf{Y}}} \in \mathbb{R}^{S * M}$
    \item $\frac{\delta L}{\delta \emph{\textbf{X}}} \in \mathbb{R}^{S * M}$
\end{enumerate}

To start, the first derivative we calculate is that of $\frac{\delta L}{\delta \gamma}$.

% DL_dgamma
% TODO: How do we get to the full form again?
{\Large $\frac{\delta L}{\delta \gamma}$ }:
\boxed{\begin{aligned}
    \frac{\delta L}{\delta \gamma}
    &=> [\frac{\delta L}{\delta \gamma}]_i
    = \frac{\delta L}{\delta \gamma}_i
    = \sum_{sj} \frac{\delta L}{\delta Y_{sj}} * \frac{\delta Y_{sj}}{\delta \gamma_i} \\
    % Now we start
    &= \sum_{sj} \frac{\delta L}{\delta Y_{sj}} 
        * \frac{\delta \gamma_j \hat{X}_{sj}}{\delta \gamma_i} \\
    &= \sum_{sj} \frac{\delta L}{\delta Y_{sj}} 
        * \delta_{ji} \hat{X}_{sj} \\
    &= \sum_{s} \frac{\delta L}{\delta Y_{si}} 
        * \hat{X}_{si} \\
    &= \sum_{s} \textbf{1}_s*  \frac{\delta L}{\delta Y_{si}} 
        * \hat{X}_{si} \\
    &=> \textbf{1}^T * [\frac{\delta L}{\delta Y} 
        \circ \hat{X}] && \text{where $$\textbf{1}$$ is a 1xS vector} \\
\end{aligned}}

\vspace{1cm}
In a similar way we can calculate the derivative with respect to $\beta$.

% DL_dbeta
% TODO: How do we get to the full form again?
{\Large $\frac{\delta L}{\delta \beta}$ }:
\boxed{\begin{aligned}
    \frac{\delta L}{\delta \beta}
    &=> [\frac{\delta L}{\delta \beta}]_i
    = \frac{\delta L}{\delta \beta}_i
    = \sum_{sj} \frac{\delta L}{\delta Y_{sj}} * \frac{\delta Y_{sj}}{\delta \beta_i} \\
    % Now we start
    &= \sum_{sj} \frac{\delta L}{\delta Y_{sj}} 
        * \frac{\delta \gamma_j \hat{X}_{sj} + \beta_J}{\delta \beta_i} \\
    &= \sum_{sj} \frac{\delta L}{\delta Y_{sj}} 
        * \delta_{ji} \\
    &= \sum_{s} \frac{\delta L}{\delta Y_{si}} \\
    &= \textbf{1}^T * \frac{\delta L}{\delta Y} && \text{where $$\textbf{1}$$ is a 1xS vector} \\
\end{aligned}}

\vspace{1cm}
Now on to calculating the derivative with respect to $X$. It would be good to first define the starting chain:

\begin{align}
    \frac{\delta L}{\delta \textbf{\emph{X}}}
    &=> \frac{\delta L}{\delta X_{ri}}
    = \sum_{sj} \frac{\delta L}{\delta Y_{sj}} * \frac{\delta Y_{sj}}{\delta X_{ri}}
\end{align}

When focusing on $\frac{\delta Y_{sj}}{\delta X_{ri}}$, there are a number of aspects that come to play. 
To start, we can break it down into a chain, where we explicitly focus on the "contribution"
that $\mu$ and $\sigma$ have on X.

\begin{align}
    \frac{\delta Y_{sj}}{\delta X_{ri}} &= 
        % Directly X first
        \frac{\delta Y_{sj}}{\delta \hat{X}_{sj}} * \frac{\delta \hat{X}_{sj}}{\delta X_{ri}}
            + \frac{\delta Y_{sj}}{\delta \mu_j} * \frac{\delta \mu_j}{\delta X_{ri}}
            + \frac{\delta Y_{sj}}{\delta \sigma^2_j} * \frac{\delta \sigma^2_j}{\delta X_{ri}}
\end{align}

This contains quite a number of steps

\end{document}